\chapter{Chapter 6. Tableaux Propositional Logic}

%\section*{4.7 Self-Study Questions}
%
%	\begin{enumerate}
%	
%		\item[4.7.1]  
%		
%	\end{enumerate}
%
\section*{6.6 Exercises}

\begin{enumerate}

	\item[6.6.1] One way of putting it is as saying that an inference is valid iff the premises together with the negation of the conclusion are unsatisfiable or inconsistent.

	\item[6.6.2]
	
	\begin{enumerate}[(a)]
	
		\item Suppose for contradiction that there is a valuation $v$ such that $\llbracket \neg (p\rightarrow q)\rrbracket_v =1$ and $\llbracket \neg (q \rightarrow p)\rrbracket_v =1$. $\llbracket \neg (p\rightarrow q)\rrbracket_v =1$ would mean that $\llbracket p\rightarrow q\rrbracket_v =0$, from which would follow that $\llbracket p\rrbracket_v =1$ and $\llbracket q\rrbracket_v =0$. (An implication is not true iff the first argument is true and the second argument is false). For $\llbracket \neg (q \rightarrow p)\rrbracket_v =1$ to hold, $\llbracket (q \rightarrow p)\rrbracket_v =0$ must be true, from which follows that $\llbracket q\rrbracket_v =1$ and $\llbracket p\rrbracket_v =0$. As we'd already established that $\llbracket p\rrbracket_v =1$ and $\llbracket q\rrbracket_v =0$, this is a contradiction. Hence such a $v$ cannot exist, so the set is unsatisfiable.
		
		\item This follows immediately from our observation 5.2.11.(i) that $\vDash p\lor\neg p$. This means that for all $v$, $\llbracket p\lor\neg p\rrbracket_v=1$. Now suppose that there is a valuation $v$ such that $\llbracket \neg (p\lor\neg p)\rrbracket_v=1$. Since $\llbracket \neg (p\lor\neg p)\rrbracket_v=1-\llbracket p\lor\neg p\rrbracket_v$, it would follow that $\llbracket p\lor\neg p\rrbracket_v=0$ in contradiction to 5.2.11.(i). Hence, there is no such valuation $v$ and $\{\neg(p\lor\neg p)\}$ is unsatisfiable.
		
	\item Suppose for contradiction that there is a valuation $v$ such that $\llbracket \neg p\rrbracket_v = 1$ and $\llbracket \neg p\rightarrow p\rrbracket_v = 1$. From $\llbracket\neg p\rrbracket_v =1$ follows that $\llbracket p\rrbracket_v=0\ (*)$. From $\llbracket \neg p\rightarrow p\rrbracket_v = 1$ follows that $max (1- \llbracket\neg p\rrbracket_v, \llbracket p\rrbracket_v) = 1$. So either $1 - \llbracket \neg p\rrbracket_v $ must be $1$ or $\llbracket p\rrbracket_v$ must be $1$. By $(*)$ follows that the latter is not the case, so $1 - \llbracket \neg p\rrbracket_v =1$ must hold. This would mean that  $\llbracket \neg p\rrbracket_v = 0 $. But we've already assumed that $\llbracket \neg p\rrbracket_v = 1$. Contradiction! Hence such a $v$ cannot exist, so the set is unsatisfiable.
	
	\item Suppose for contradiction that there is a $v$ such that $\llbracket\neg p\rrbracket_v=1$ and $\llbracket(p\to q)\to p\rrbracket_v=1$. Since $\llbracket\neg p\rrbracket_v=1$ and $\llbracket\neg p\rrbracket_v=1-\llbracket p\rrbracket_v$, it follows that $\llbracket p\rrbracket_v=0$. Now consider $\llbracket(p\to q)\to p\rrbracket_v$. We know that $\llbracket(p\to q)\to p\rrbracket_v=max(1-\llbracket p\to q\rrbracket_v, \llbracket p\rrbracket)=max(1-max(1-\llbracket p\rrbracket_v,\llbracket q\rrbracket_v), \llbracket p\rrbracket)$. Since $\llbracket p\rrbracket_v=0$, we can infer that $max(1-max(1-\llbracket p\rrbracket_v,\llbracket q\rrbracket_v), \llbracket p\rrbracket)=max(1-max(1-0,\llbracket q\rrbracket_v),0)=max(1-max(1,\llbracket q\rrbracket_v),0)=max(1-1,0)-max(0,0)=0$, in contradiction to $\llbracket(p\to q)\to p\rrbracket_v=1$. Hence there can be no such $v$ and our set is unsatisfiable.
	
	
\end{enumerate}

	\item[6.6.3] We need to show two things: (a) if $\vDash \neg (\phi_1\land\mathellipsis\land\phi_n)$, then $\{\phi_1, \mathellipsis,\phi_n\}$ is unsatisfiable, and (b) if $\{\phi_1, \mathellipsis,\phi_n\}$ is unsatisfiable, then $\vDash \neg (\phi_1\land\mathellipsis\land\phi_n)$. We do so in turn:
	
	\begin{enumerate}[(a)]
	
		\item Suppose that $\vDash \neg (\phi_1\land\mathellipsis\land\phi_n)$. This means that for each $v$, $\llbracket \neg (\phi_1\land\mathellipsis\land\phi_n)\rrbracket_v=1$. Since $\llbracket \neg (\phi_1\land\mathellipsis\land\phi_n)\rrbracket_v=1-\llbracket (\phi_1\land\mathellipsis\land\phi_n)\rrbracket_v$, we can infer that $\llbracket (\phi_1\land\mathellipsis\land\phi_n)\rrbracket_v=0$. Now consider $\llbracket (\phi_1\land\mathellipsis\land\phi_n)\rrbracket_v$. We can easily show that $\llbracket (\phi_1\land\mathellipsis\land\phi_n)\rrbracket_v=min(\llbracket \phi_1\rrbracket_v, \mathellipsis, \llbracket\phi_n\rrbracket_v)$ (exercise, proof this by induction on natural numbers). Since $min(\llbracket \phi_1\rrbracket_v, \mathellipsis, \llbracket\phi_n\rrbracket_v)=0$, it follows that some $\phi_i$ for $1\leq i\leq n$ must be such that $\llbracket\phi_i\rrbracket_v=0$. But that just means that for each valuation $v$, some $\phi_i$ must be false. Hence there can be no valuation that makes all the members of $\{\phi_1, \mathellipsis,\phi_n\}$ true, which is what we needed to show.
		
		\item Suppose that $\{\phi_1, \mathellipsis,\phi_n\}$ is unsatisfiable, i.e. there is no valuation that that makes all the members of $\{\phi_1, \mathellipsis,\phi_n\}$ true. We wish to derive that $\vDash \neg (\phi_1\land\mathellipsis\land\phi_n)$, i.e. for all valuations $v$, $\neg (\phi_1\land\mathellipsis\land\phi_n)$ is true. So let $v$ be arbitrary. Since $\{\phi_1, \mathellipsis,\phi_n\}$ is unsatisfiable, it follows that  that some $\phi_i$ for $1\leq i\leq n$ must be such that $\llbracket\phi_i\rrbracket_v=0$. Now consider $\llbracket (\phi_1\land\mathellipsis\land\phi_n)\rrbracket_v$. We've already observed that $\llbracket (\phi_1\land\mathellipsis\land\phi_n)\rrbracket_v=min(\llbracket \phi_1\rrbracket_v, \mathellipsis, \llbracket\phi_n\rrbracket_v)$. Since there is a $\phi_i$ such that $\llbracket\phi_i\rrbracket_v=0$, it follows that $min(\llbracket \phi_1\rrbracket_v, \mathellipsis, \llbracket\phi_n\rrbracket_v)=0$. Since $\llbracket \neg (\phi_1\land\mathellipsis\land\phi_n)\rrbracket_v=1-\llbracket (\phi_1\land\mathellipsis\land\phi_n)\rrbracket_v$, it immediately follows that $\llbracket \neg (\phi_1\land\mathellipsis\land\phi_n)\rrbracket_v=1$, as desired.
	
	\end{enumerate}
	
	\item[6.6.4] 
	
	\begin{enumerate}[(a)]

\item $p\to q, r\to q\vdash (p\lor r)\to q$

\begin{center}
\begin{prooftree}
{
line numbering=false,
for tree={s sep'=10mm},
single branches=true,
close with=\xmark
}
[p\to q, grouped [r\to q, grouped [\neg ((p\lor r)\to q), grouped [p\lor r [\neg q [\neg p [p, close] [r [\neg r, close] [q, close]]] [q, close]]]]]]
\end{prooftree}
\end{center}

\item $p\to (q\land r), \neg r\vdash \neg p$

\begin{center}
\begin{prooftree}
{
line numbering=false,
for tree={s sep'=10mm},
single branches=true,
close with=\xmark
}
[p\to (q\land r), grouped [\neg r, grouped [\neg\neg p, grouped  [p [\neg p, close] [q\land r [q[r,close]]]]]]]
\end{prooftree}
\end{center}

\item $((p\to q)\to q)\to q$

\begin{center}
\begin{prooftree}
{
line numbering=false,
for tree={s sep'=10mm},
single branches=true,
close with=\xmark
}
[\neg (((p\to q)\to q)\to q) [((p\to q)\to q) [\neg q [\neg (p\to q) [p [\neg q]]] [q, close]]]]
\end{prooftree}
\end{center}

Counter-model: $v(p)=1, v(q)=0$. 

\newpage

\item $\vdash ((p\to q)\land (\neg p\to q))\to \neg p$
{\footnotesize
\begin{center}
\begin{prooftree}
{
line numbering=false,
for tree={s sep'=10mm},
single branches=true,
close with=\xmark
}
[\neg (((p\to q)\land (\neg p\to q))\to \neg p) 
	[((p\to q)\land (\neg p\to q)) 
		[\neg \neg p 
			[p 
				[p\to q
					[\neg p\to q
						[\neg p, close]
						[q
							[\neg\neg p [p]]
							[q]
							]
						]
					]
				]
			]
		]	
]				
\end{prooftree}
\end{center}

Counter-model: $v(p)=1, v(q)=1$
}

\item $p\leftrightarrow (q\leftrightarrow r)\vdash (p\leftrightarrow q)\leftrightarrow r$

{\footnotesize\begin{center}
\begin{prooftree}
{
line numbering=false,
for tree={s sep'=5mm},
single branches=true,
close with=\xmark
}
[p\leftrightarrow (q\leftrightarrow r), grouped [\neg((p\leftrightarrow q)\leftrightarrow r),grouped 
[p [(q\leftrightarrow r)
	[(p\leftrightarrow q) [\neg r
		[q [r,close] ]
		[\neg q [\neg r
			[p [q, close]]
			[\neg p [\neg q, close]]
			]]
		]]
	[\neg (p\leftrightarrow q) [r
		[q [r
			[p [\neg q, close]]
			[\neg p [q, close]]
			]]
		[\neg q [\neg r,close] ]
		]] 
	]
	]
[\neg p [\neg(q\leftrightarrow r)
	[(p\leftrightarrow q) [\neg r
		[q [\neg r
			[p [q, close]]
			[\neg p [\neg q, close]]
			]]
		[\neg q [r,close] ]
		]]
	[\neg (p\leftrightarrow q) [r
		[q [\neg r,close] ]
		[\neg q [r
			[p [\neg q, close]]
			[\neg p [q,close]]
			]]
		]] 
	]
	]
]
]
]
\end{prooftree}
\end{center}}

\newpage

\item $\neg(p\to q)\land \neg(p\to r)\vdash \neg q\lor \neg r$

\begin{center}
\begin{prooftree}
{
line numbering=false,
for tree={s sep'=5mm},
single branches=true,
close with=\xmark
}
[\neg(p\to q)\land \neg(p\to r), grouped [\neg(\neg q\lor \neg r), grouped
	[\neg(p\to q) [\neg(p\to r)
		[\neg\neg q [\neg \neg r
			[p [\neg q, close]]
			]]
		]
	]
]
]
\end{prooftree}
\end{center}

\item $p\land (\neg r\lor s), \neg (q\to s)\vdash r$

\begin{center}
\begin{prooftree}
{
line numbering=false,
for tree={s sep'=5mm},
single branches=true,
close with=\xmark
}
[p\land (\neg r\lor s), grouped  [\neg (q\to s), grouped [\neg r, grouped
	[p [(\neg r\lor s) [q [\neg s
	[\neg r]
	[s, close]
	]]]] 
]
]
]
\end{prooftree}
\end{center}

Counter-model: $v(p)=1, v(q)=1, v(r)=0, v(s)=0$.

\newpage

\item $\vdash (p\to (q\to r))\to (q\to (p\to r))$

\begin{center}
\begin{prooftree}
{
line numbering=false,
for tree={s sep'=5mm},
single branches=true,
close with=\xmark
}
[\neg((p\to (q\to r))\to (q\to (p\to r)))
	[(p\to (q\to r)) [\neg(q\to (p\to r))
	[q [\neg (p\to r)
	[p [\neg r
		[\neg p, close]
		[(q\to r)
			[\neg q, close]
			[r, close]
			]
		]
	]
	]
	]
	]
	]
]
\end{prooftree}
\end{center}

\item $\neg(p\land \neg q)\lor r, p\to (r\leftrightarrow s)\vdash p\leftrightarrow q$

(See last page. This was a tough one \smiley)

\begin{sidewaysfigure}[h!]
\begin{prooftree}
{
line numbering=false,
for tree={s sep'=10mm},
single branches=true,
close with=\xmark
}
[\neg(p\land \neg q)\lor r, grouped [p\to (r\leftrightarrow s), grouped [\neg(p\leftrightarrow q), grouped
	[\neg(p\land \neg q)
		[\neg p
			[p [\neg q, close]]
			[\neg p [q
				[\neg p]
				[\neg\neg q [q]]
				]]
			]
		[(r\leftrightarrow s)
			[p [\neg q
				[\neg p, close]
				[\neg\neg q, close]
				]]
			[\neg p [q
				[\neg p
					[r [s]]
					[\neg r [\neg s]]
					]
				[\neg\neg q [q
					[r [s]]
					[\neg r [\neg s]]
					]]
				]]
			]
		]
	[r
		[\neg p
			[p [\neg q, close]]
			[\neg p [q]]
			]
		[(r\leftrightarrow s)
			[p [\neg q
				[r [s]]
				[\neg r [\neg s, close]]
				]]
			[\neg p [q
				[r [s]]
				[\neg r [\neg s, close]]
				]]
			]
		]
	]
]
]
\end{prooftree}\\[2ex]

Counter-model $v(p)=0, v(q)=1, v(r)=0$.

\end{sidewaysfigure}

\newpage

\item $p\leftrightarrow \neg\neg q, \neg q\to (r\land \neg s), s\to (p\lor q)\vdash (s\land q)\to p$

(Note: Here I made a short tableaux by strategically applying the rules in a certain order. You might get another tableaux if you use the rules in different order.)

\begin{center}
\begin{prooftree}
{
line numbering=false,
for tree={s sep'=5mm},
single branches=true,
close with=\xmark
}
[p\leftrightarrow \neg\neg q, grouped [\neg q\to (r\land \neg s), grouped [s\to (p\lor q), grouped [\neg((s\land q)\to p), grouped
	[s\land q [\neg p [s [q
		[\neg s, close]
		[p\lor q
			[p, close]
			[q
				[p [\neg\neg q, close]]
				[\neg p [\neg\neg\neg q [\neg q, close]]]
				]
			]
		]
		] 
	]
	]
]
]
]
]
\end{prooftree}
\end{center}

\end{enumerate} 

\end{enumerate}

	
%%% Local Variables: 
%%% mode: latex
%%% TeX-master: "../../logic.tex"
%%% End: 