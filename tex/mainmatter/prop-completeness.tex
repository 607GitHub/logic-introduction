\chapter{Soundness and Completeness}

\emph{This chapter is rather short, but it packs a punch: it contains two  relatively complicated proofs. We're going to spend one entire lecture going through the details.}

\section{Soundness and Completeness}

\begin{enumerate}[\thesection.1]

		\item Recall from the introduction that we want our proof system to be such that we can derive the conclusion from the premises in all and only the valid inferences. In this chapter, we set out to prove that our tableau system for propositional logic enjoys this property: we set out to prove that our tableau system is \emph{sound and complete}. Remember from the introduction that soundness means that \emph{only} in valid inferences, we can derive the conclusion from the premises, while completeness means that in \emph{all} the valid inferences, we can derive the conclusion from the premises. Using our official notation, we can now state the two theorems we wish to prove as follows:
		\begin{description}
		
			\item[Soundness Theorem.] If $\Gamma\vdash\phi$, then $\Gamma\vDash\phi$.
			
			\item[Completeness Theorem.] If $\Gamma\vDash\phi$, then $\Gamma\vdash\phi$.
		
		\end{description}
We're going to prove these two theorems in turn, beginning with soundness. But before, let us make a couple of remarks about soundness and completeness results in general.

	\item The reason why having a sound and complete proof system is desirable is that it allows us to approach validity in a purely syntactic fashion. Remember that proof systems are purely syntactic, they only manipulate formulas without reference to the semantic clauses. If we have a sound and complete proof system, this means that even though we don't explicitly talk about semantics in our system, we still effectively capture the semantically defined notion of validity---no small feat! From an AI perspective what's neat about this is that having reduced validity to syntactic derivability makes establishing validity \emph{much} more tractable for computer systems.
	
	\item Of the two kinds of theorems, soundness and completeness, the former is typically easier to show than the letter. Intuitively, the soundness theorem is a kind of ``sanity check'' for our proof system. As we stated it above, the theorem states that only in valid inferences, we can derive the conclusion from the premises. Why is that a sanity check? Well, because it means that if we can derive something, then it follows---it can't happen that we derive something and it doesn't follow.
	
	\item The completeness theorem is typically (much) harder to prove. And without knowing how the proof goes, it's already possible to see why. The theorem states that \emph{every} conclusion that can validly be drawn from a set of premises can be derived from them. But surely we usually don't know \emph{all} conclusions that can validly be drawn from a set of premises: there are many (actually infinitely many) of them, it takes some time to figure that out. But the completeness theorem states that, even if we don't know which are the conclusion we can validly draw, we can be sure that we can derive them. That's surprising! (I hope \dots) 
	
	\item The tableau system we use in this paper has the nice feature that its soundness proof \emph{and} its completeness proof are relatively perspicuous. This is why we can give them in an introductory course like this. The completeness proof for the Hilbert or natural deduction calculi for propositional logic, for example, is much harder (although ultimately based on the same ideas). The aim of this chapter is two-fold: first, I want to introduce you to the idea of soundness and completeness proofs, the kinds of things you have to do to establish a result like this, etc.; and second, I want to set you up for the beginning of the second part of the course, in which we're going to focus more on proving things in logic---so why not begin with one of the most exciting theorems you can prove \smiley

	\item Before we get started, let's briefly say something about the proof \emph{idea}. Remember that (in 6.3.3), we said that the idea behind the tableau rules is are the following two properties:
		\begin{description}
			
				\item[Down Preservation.] If the formula at the parent node of a rule is true under a valuation, then at least one formula on a newly generated child node is true under the valuation.
				
				\item[Up Preservation.] If a formula at a newly generated child node is true under a valuation, then the formula at the parent node is true.
				
		\end{description}
It turns out that these two properties, when thought through carefully lead to the desired results: down preservation leads to soundness and up preservation leads to completeness. Effectively, what down and up preservation together guarantee is that for each rule, we're thinking through precisely the ways in which a formula can be true: down preservation means that we're considering \emph{all} the possibilities of the formula being true, und up preservation means that we're considering \emph{only} possibilities of the formula being true. We'll have to put in some work, but that's the essence of it. To prepare yourself for the proof, remind yourself of the rules and check that they do indeed have the two properties just described:	
	
	
			\begin{center}
					
					\begin{prooftree}
					{
					line numbering=false,
					line no sep= 2cm,
					for tree={s sep'=5mm},
					single branches=true,
					close with=\xmark
					}
					[\neg\neg \phi [\phi ] ]
					\end{prooftree}
					%
					\begin{prooftree}
					{
					line numbering=false,
					line no sep= 2cm,
					for tree={s sep'=5mm},
					single branches=true,
					close with=\xmark
					}
					[\phi\land\psi [\phi [\psi ] ] ]
					\end{prooftree}
					%
					\begin{prooftree}
					{
					line numbering=false,
					line no sep= 2cm,
					for tree={s sep'=5mm},
					single branches=true,
					close with=\xmark
					}
					[\neg (\phi\land\psi) [\neg \phi ] [\neg \psi ] ]
					\end{prooftree}
					%
					\begin{prooftree}
					{
					line numbering=false,
					line no sep= 2cm,
					for tree={s sep'=5mm},
					single branches=true,
					close with=\xmark
					}
					[\phi\lor\psi [\phi ] [\psi ] ]
					\end{prooftree}
					%
					\begin{prooftree}
					{
					line numbering=false,
					line no sep= 2cm,
					for tree={s sep'=5mm},
					single branches=true,
					close with=\xmark
					}
					[\neg(\phi\lor\psi) [\neg\phi [\neg\psi ] ] ]
					\end{prooftree}

					\vspace{2ex}

					\begin{prooftree}
					{
					line numbering=false,
					line no sep= 2cm,
					for tree={s sep'=5mm},
					single branches=true,
					close with=\xmark
					}
					[\neg (\phi\to\psi) [\phi [\neg \psi ] ] ]
					\end{prooftree}
					%
					\begin{prooftree}
					{
					line numbering=false,
					line no sep= 2cm,
					for tree={s sep'=5mm},
					single branches=true,
					close with=\xmark
					}
					[\phi\to\psi [\neg \phi ] [\psi ] ]
					\end{prooftree}
					%
					\begin{prooftree}
					{
					line numbering=false,
					line no sep= 2cm,
					for tree={s sep'=5mm},
					single branches=true,
					close with=\xmark
					}
					[\phi\leftrightarrow \psi [\phi [\psi] ] [\neg \phi [\neg \psi] ] ]]
					\end{prooftree}
					%
					\begin{prooftree}
					{
					line numbering=false,
					line no sep= 2cm,
					for tree={s sep'=5mm},
					single branches=true,
					close with=\xmark
					}
					[\neg(\phi\leftrightarrow \psi) [\phi [\neg \psi] ] [\neg \phi [ \psi] ] ]]
					\end{prooftree}

				\end{center}
		

\end{enumerate}	

\section{The Soundness Theorem}

\begin{enumerate}[\thesection.1]
		
  \item In this section, we're aiming to prove the soundness theorem, i.e. the fact that if
	$\Gamma\vdash \phi$,
	then
	$\Gamma\vDash\phi$.
	Actually, what we're going to prove is the contrapositive (remember contrapositive proof from \S2): if
	$\Gamma\nvDash \phi$,
	then
	$\Gamma\nvdash\phi$.
	Let's set out our strategy: First, remember that
	$\Gamma\nvdash \phi$
	means that the tableau for
	$\Gamma\cup\{\neg\phi\}$
	is open, i.e. at least one branch in the tableau doesn't contain both some $p$ and $\neg p$ (6.3.8).
	So, in order to obtain our result---that if
	$\Gamma\nvDash\phi$,
	then
	$\Gamma\nvdash\phi$
	---we need to show that we can derive from
	$\Gamma\nvDash\phi$
	that at least one branch in the tableau for
	$\Gamma\cup\{\neg\phi\}$
	doesn't contain both some $p$ and $\neg p$.
	How can we achieve this?
	Well, remember that
	$\Gamma\nvDash \phi$
	means that there's a valuation $v$ such that
	$\llbracket \psi\rrbracket_v=1$
	for all
	$\psi\in\Gamma$
	and
	$\llbracket\phi\rrbracket_v=0$
	(cf. 5.2.8).
	So, we can use the information that this countermodel exists.
	Now this is where the down preservation property comes into play.
	Note that the initial list---in our case,
	$\Gamma\cup\{\neg\phi\}$
	---is at the root of our the tableau.
	Our countermodel showing
	$\Gamma\nvDash \phi$
	makes all the members of
	$\Gamma\cup\{\neg\phi\}$
	true.
	And by the down preservation property, whenever we apply a rule to our initial list, at least one branch contains a true formula in our countermodel.
	So, our final tableau must contain at least one branch such that all the formulas on that branch are true in our countermodel.
	But then that branch can't contain both $p$ and $\neg p$, since the two cannot both be true under any valuation.
	Hence the branch must be open. This is how we're going to prove soundness, so let's get to work.
		
	\item In the following, we will talk about tableaux as the kinds of trees constructed according to the rules laid out in 6.3.2. Remember that a tableau is \emph{complete} if every rule that can be applied has been applied; otherwise we say that the tableau is \emph{in}complete. Now suppose that $v$ is a valuation and $B$ a branch of a (possibly incomplete) tableau. Then we say that $v$ is \emph{faithful} to $B$ iff $\llbracket\phi\rrbracket_v=1$, for all $\phi\in B$, i.e. $v$ is faithful to $B$ iff under $v$ all the formulas on $B$ are true. Note that the associated interpretation $v_b$ of an open branch $B$ in a complete tableau (6.3.5) is a paradigm example of a faithful valuation (though we haven't proven this yet in generality): $v_B$ is faithful to $B$. So every countermodel produced by the tableau method in 6.3.11 is a (paradigm) example of a faithful interpretation for the open branch it was derived from. It's worth convincing yourselves of this fact in order to understand what's about to happen next. So go ahead and check that in each case in 6.3.11, $v_B$ makes all the members of $B$ true. The concept of faithfulness will be the central concept in our soundness and completeness proof.
			
	\item Note that if we have an incomplete tableau and apply a rule to some formula in it, each branch of our incomplete tableau will be extended with new formulas (this is what it means to properly apply a rule to an incomplete tableau, cf. 6.3.2.2). Our central lemma, which will lead to the soundness theorem, states that by extending branches in this way, we preserve faithfulness of valuations:
	\begin{lemma}[Soundness Lemma]
	Let $v$ be a valuation that is faithful to a branch $B$ of an incomplete tableau. If a rule is applied to a formula in the tableau, then $v$ is faithful to at least one branch $B'$ which extends $B$ in the new tableau.
	\end{lemma}
This lemma is, in a sense, a more precise version of the down preservation property. Before we set out to prove it, let's consider an example to see what the lemma says. Let's do the tableau for $\{p\lor q, \neg p \lor \neg q\}$ step-by-step and consider the faithful valuation along the way (assuming $\mathcal{P}=\{p,q\}$). We begin with the initial list:
	\begin{center}
		\begin{prooftree}
					{
					line numbering=false,
					line no sep= 1cm,
					for tree={s sep'=5mm},
					single branches=true,
					close with=\xmark
					}
					[p\lor q, grouped [\neg p\lor \neg q, grouped ] ]
					\end{prooftree}
		\end{center}
	The initial list is a limit-case of a tableau, one with only one node and one branch. It's easily checked that there are precisely two valuations that are faithful to this branch (which consists solely of the initial list): $v_1$ with $v_1(p)=1$ and $v_1(q)=0$ and $v_2$ with $v_2(p)=0$ and $v_2(q)=1$. Now let's begin constructing our tableau by applying the rule for $p\lor q$:
		\begin{center}
		\begin{prooftree}
					{
					line numbering=false,
					line no sep= 1cm,
					for tree={s sep'=5mm},
					single branches=true,
					close with=\xmark
					}
					[p\lor q, grouped [\neg p\lor \neg q, grouped [p] [q]  ] ]
					\end{prooftree}
		\end{center}
	We now have two branches in our tableau: $B_1$ which contains $p, \neg p\lor \neg q,$ and $p\lor q$ and the initial list and $B_2$ which contains $q, \neg p\lor \neg q,$ and $p\lor q$. Now it's easily checked that $v_1$ remains faithful to at least one of the new branches created, namely $B_1$: $v_1$ makes $p$ true. The valuation $v_1$, of course, is not faithful to \emph{all} the new branches, $B_2$ contains $q$ and $v_1$ makes $q$ false. But our lemma states that $v_1$ needs to make all the formulas on \emph{one} of the new branches true. (The case for $v_2$ is completely analogous). Continuing constructing our tableau hopefully drives the point home:  
		\begin{center}
		\begin{prooftree}
					{
					line numbering=false,
					line no sep= 1cm,
					for tree={s sep'=5mm},
					single branches=true,
					close with=\xmark
					}
					[p\lor q, grouped [\neg p\lor \neg q, grouped [p [\neg p, close] [\neg q] ] [q [\neg p] [\neg q, close] ]  ] ]
					\end{prooftree}
		\end{center}	
	We now have \emph{four} branches in our tableau:
	\begin{itemize}
	
		\item $B_1^1$ with $\neg p, p, \neg p\lor \neg q, p\lor q$
		\item $B_1^2$ with $\neg q, p, \neg p\lor \neg q, p\lor q$
		\item $B_2^1$ with $\neg p, q, \neg p\lor \neg q, p\lor q$
		\item $B_2^2$ with $\neg q, q, \neg p\lor \neg q, p\lor q$
	
	\end{itemize}
	Of the two branches extending $B_1$, namely $B_1^1$ and $B_1^2$, $v_1$ is faithful again to one of them: $B_1^2$. Note that there is no valuation faithful to $B_1^1$, since the branch is closed: we have $p,\neg p\in B_1^1$ and there can't be a valuation that makes both $p$ and $\neg p$ true. So, starting with a valuation ($v_1$) that made the members of our initial list true, by keeping track of that valuation throughout the construction of our tableau, we ultimately found a branch in the final, complete tableau ($B_1^2$) such that the valuation makes all the formulas on that branch true---all thanks to the fact that we could always find at least one new branch with a true formula on it.

\item We're now going to prove that this example generalizes, we're going to prove our lemma:
	\begin{proof}
	The proof consists in a one-by-one inspection of the rules. There are 9 rules, so 9 cases. Here I'm not going to exercise all of the cases for you, I'll leave some work for you (exercise 7.6.1). I will do the cases for (a) the rule for $\phi\to\psi$ and (b) the rule for $\neg(\phi\to \psi)$.
	
	\begin{enumerate}[(a)]
					
		\item Suppose that $v$ is a valuation faithful to branch $B$ of some incomplete tableau. Suppose further that $\phi\to\psi\in B$ and now the rule 
		
					\begin{center}
					\begin{prooftree}
					{
					line numbering=false,
					line no sep= 2cm,
					for tree={s sep'=10mm},
					single branches=true,
					close with=\xmark
					}
					[\phi\to\psi [\neg \phi ] [\psi ] ]
					\end{prooftree}
					\end{center}
					
			is applied, extending the branch accordingly. This means that we have two new branches extending $B$, $B_1$ and $B_2$. And we have $B_1=B\cup\{\neg\phi\}$ and $B_2=B\cup\{\psi\}$. We already know that $v$ is faithful to $B$, and so $\llbracket\phi\to\psi\rrbracket_v=1$, in particular. Since $\llbracket\phi\to\psi\rrbracket_v=max(1-\llbracket\phi\rrbracket_v,\llbracket\psi\rrbracket_v)$, it follows that either $\llbracket\phi\rrbracket_v=0$ or $\llbracket\psi\rrbracket_v=1$.So, we can distinguish two cases:
	\begin{itemize}
		
			\item In the first case, $\llbracket\neg\phi\rrbracket_v=1-\llbracket\phi\rrbracket_v=1$. But since $v$ is already faithful to $B$, this means that $v$ is faithful to $B_1=B\cup\{\neg\phi\}$. 
		
			\item In the second case, since $v$ is already faithful to $B$, we immediately get that $v$ is faithful to $B_2=B\cup\{\psi\}$. 
	
		\end{itemize}
	So, either way, $v$ is faithful to at least one new branch created by the rule for $\phi\to\psi$, which is what we needed to show.
	
	\item For the second case, suppose that $v$ is a valuation faithful to branch $B$ of some incomplete tableau and that $\neg(\phi\to\psi)\in B$. Now the rule 
		\begin{center}{
					\begin{prooftree}
					{
					line numbering=false,
					line no sep= 2cm,
					for tree={s sep'=10mm},
					single branches=true,
					close with=\xmark
					}
					[\neg(\phi\to\psi) [\phi [\neg\psi ] ] ]
					\end{prooftree}}
					\end{center}
					
			is applied, which gives us one new branch $B'=B\cup\{\phi,\neg\psi\}$. Since $v$ is faithful to $B$ and $\neg(\phi\to\psi)\in B$, we get that $\llbracket\neg(\phi\to\psi)\rrbracket_v=1$. From this, since $\llbracket\neg(\phi\to\psi)\rrbracket_v=1-max(1-\llbracket\phi\rrbracket_v, \llbracket\psi\rrbracket_v)$, we get that $max(1-\llbracket\phi\rrbracket_v, \llbracket\psi\rrbracket_v)=0$. So $\llbracket\phi\rrbracket_v=1$ and $\llbracket\psi\rrbracket_v=0$. Since $\llbracket\neg\psi\rrbracket_v=1-\llbracket\psi\rrbracket_v$, we can conclude  $\llbracket\neg\psi\rrbracket_v=1$. But now, since $v$ was faithful to $B$, $B'=B\cup\{\phi,\neg\psi\}$, and $\llbracket\phi\rrbracket_v=1$ as well as $\llbracket\neg\psi\rrbracket_v=1$, we get that $v$ is faithful to $B'$.
			
	\end{enumerate}
	The remaining cases work similarly and you should work them out yourself.
	\end{proof}
	
	\item We will now use the soundness lemma to conclude the soundness \emph{theorem}:
	\begin{theorem}[Propositional Soundness]
	If $\Gamma\vdash\phi$, then $\Gamma\vDash\phi$.
	\end{theorem}
	\begin{proof}
	We follow the proof strategy laid out in 7.2.1. We prove the contrapositive that if $\Gamma\nvDash\phi$, then $\Gamma\nvdash\phi$. So, suppose that $\Gamma\nvDash\phi$. So there's a valuation, $v$, such that $\llbracket \psi\rrbracket_v=1$ for all $\psi\in\Gamma$ and $\llbracket\phi\rrbracket_v=0$. We now successively construct the tableau for $\Gamma\cup\{\neg\phi\}$ and prove that it must be open. First, we write down the initial list $\Gamma\cup\{\neg\phi\}$. Note that since $v$ is such that $\llbracket \psi\rrbracket_v=1$ for all $\psi\in\Gamma$ and $\llbracket\phi\rrbracket_v=0$, it follows immediately that $v$ is faithful to the only branch in the (incomplete) tableau consisting only of the initial list. We now successively apply the rules to turn our initial list into a complete tableau. Every time we apply a rule, by our soundness lemma 7.2.3, we get at least one branch that $v$ is faithful to. Hence $v$ must be faithful to at least one branch in the complete tableau. Call this branch $B$.
	
	We now conclude that $B$ cannot be closed. We show this indirectly. Suppose that $B$ was closed. Then there would exists a $p\in\mathcal{P}$ such that both $p\in B$ and $\neg p\in B$. Since $v$ is faithful to $B$, this would mean that $\llbracket p\rrbracket_v=1$ and $\llbracket\neg p\rrbracket_v=1$. But this just means that $v(p)=1$ and $v(p)=0$, which is impossible. Hence $B$ cannot be closed. 
	
	But if $B$ cannot be closed, then $B$ must be open. Since a tableau is open iff at least one branch in the tableau is open (6.3.2.4), we conclude that our complete tableau must be open. Hence $\Gamma\nvdash\phi$, by definition, which is what we needed to show.
	
	\end{proof}
	
	\item Before we conclude the section and move to completeness, we remark that what we've just prove can also be interpreted differently: effectively, what we've proven is that the tableau method, construed as the algorithm laid out in 6.3.2, gives the correct result for unsatisfiable sets:
	\begin{theorem}[Tableau Verification Part 1]
	If the algorithm laid out in 6.3.2 gives the answer that a set is unsatisfiable, then the set \emph{is} unsatisfiable. 
	\end{theorem}
	\begin{proof}
	Suppose that the tableau method gives the result that a set $\Gamma$ is unsatisfiable. By 6.3.2.4, this means that the tableau must be closed. Now suppose, for proof by contradiction, that the set $\Gamma$ is satisfiable. This means, by definition, that there's a valuation $v$ that makes all the members of $\Gamma$ true. By the same reasoning as in the proof of Soundness, we can conclude that there's at least one open branch in the complete tableau for $\Gamma$. Hence the tableau must be open, in contradiction to our assumption that the tableau method gave the result that $\Gamma$ is unsatisfiable. Hence $\Gamma$ must indeed be unsatisfiable, which is what we needed to show. 
	\end{proof}

\end{enumerate}

\section{The Completeness Theorem}

\begin{enumerate}[\thesection.1]

  \item We now move to completeness.
	There is a sense in which the completeness proof is easier than the soundness proof, namely that the proof \emph{idea} is easier:
	we simply prove that the associated valuation for an open branch in a complete tableau (cf. 6.3.5) is faithful to that branch.
	From this, we can quickly conclude completeness, again by contrapositive reasoning.
	For suppose that
	$\Gamma\nvdash\phi$.
	This means, by definition, that the tableau for
	$\Gamma\cup\{\neg\phi\}$ is open.
	Hence we get an open branch and associated interpretation, which would then make all the members of
	$\Gamma\cup\{\neg\phi\}$
	true and thus show that
	$\Gamma\nvDash\phi$.
	So, this gives us that if
	$\Gamma\nvdash\phi$,
	then
	$\Gamma\nvDash\phi$, which is just the contrapositive of completeness.
	The devil, you rightly suspect, lies of course in the detail: proving that the associated interpretation is indeed faithful.
	That's what we will now do, giving us the completeness lemma.
	In preparation for the proof, remind yourself of the definition of the associated interpretation:
	\[v_B(p)=\begin{cases} 1 &\text{if }p\in B\\0&\text{if }p\notin B\end{cases}\]
	And make sure that you actually did check the examples in 6.3.11, as I asked you to in 7.2.2.
	
	\item We prove:
	\begin{lemma}[Completeness Lemma]
	Let $B$ be an open branch of a complete tableau. Then $v_B$ is faithful to $B$.
	\end{lemma}
	The main proof is a (rather complicated) induction, so make sure that you're familiar with the proof principle (\S4.2). Ready? Here we go:
	\begin{proof}
	Let $B$ be an open branch of a complete tableau and $v_B$ its associated valuation. First, we split up our claim into two parts. We note that $v_B$ is faithful to $B$, i.e. $\llbracket\phi\rrbracket_v=1$ for all $\phi\in B$, iff for all $\phi\in\mathcal{L}$:
	\begin{enumerate}[1.]
	
		\item if $\phi\in B$, then $\llbracket\phi\rrbracket_{v_B}=1$, and 
		\item if $\neg \phi\in B$, then $\llbracket\phi\rrbracket_{v_B}=0$.
	
	\end{enumerate}
	So, what we're going to prove is the conjunction of (a) and (b). We're going to do this by induction.
	
	\begin{enumerate}[(i)]
	
		\item \emph{Base case}. We need to show that for all $p\in \mathcal{P}$, 1. if $p\in B$, then $\llbracket p\rrbracket_{v_B}=1$, and 2. if $\neg p\in B$, then $\llbracket p\rrbracket_{v_B}=0$. Claim 1. is immediate by Definition 6.3.5. To see that 2. holds, note that if $\neg p\in B$, then it cannot be that $p\in B$. Because then $p,\neg p\in B$ and so $B$ would be closed contrary to our assumption that it's open. But by definition, if $p\notin B$, we have $v_B(p)=0$. and since $\llbracket p\rrbracket_{v_B}=v_B(p)$, we get our desired claim.
		
		\item \emph{Induction steps}. Now, things get serious:
		
		\begin{enumerate}[(a)]
		
			\item We need to prove that if $\phi$ enjoys the property, then $\neg \phi$ enjoys the property. In our case, this means that assuming
		\begin{enumerate}[1.]
	
		\item if $\phi\in B$, then $\llbracket\phi\rrbracket_{v_B}=1$, and 
		\item if $\neg \phi\in B$, then $\llbracket\phi\rrbracket_{v_B}=0$.
	
	\end{enumerate}
	as our induction hypothesis, we need to derive that 
		\begin{enumerate}[1'.]
	
		\item if $\neg\phi\in B$, then $\llbracket\neg\phi\rrbracket_{v_B}=1$, and 
		\item if $\neg\neg \phi\in B$, then $\llbracket\neg\phi\rrbracket_{v_B}=0$.\footnote{Note the double negation here. This is \emph{not} a typo!}
	
	\end{enumerate}
	We do so in turn:
	
	\begin{itemize}
	
		\item First, 1'. Suppose that $\neg \phi\in B$. By induction hypothesis 2., this means that $\llbracket\phi\rrbracket_{v_B}=0$. But $\llbracket\neg\phi\rrbracket_{v_B}=1-\llbracket\phi\rrbracket_{v_B}$ and so $\llbracket\neg\phi\rrbracket_{v_B}=1$, as desired.
	
		\item For 2'. Assume that $\neg\neg \phi\in B$. Since $B$ is an open branch of a \emph{complete} tableau, every rule that can be applied has been applied. And so, the rule for $\neg\neg\phi$ has been applied:
		\begin{center}
					\begin{prooftree}
					{
					line numbering=false,
					line no sep= 2cm,
					for tree={s sep'=10mm},
					single branches=true,
					close with=\xmark
					}
					[\neg\neg\phi [\phi ] ]
					\end{prooftree}
			\end{center} 
		So, it must be the case that $\phi\in B$. But then, by induction hypothesis 1., we get that $\llbracket\phi\rrbracket_{v_B}=1$. And since $\llbracket\neg\phi\rrbracket_{v_B}=1-\llbracket\phi\rrbracket_{v_B}$, we get $\llbracket\neg\phi\rrbracket_{v_B}=0$, as desired.

		
		\end{itemize}
		
		\item We have a different sub-case for $\circ=\land,\lor,\to,\leftrightarrow$. I will go through the case for $\circ=\land$ to illustrate the idea and leave the remaining cases as an exercise (7.6.2).
		
		We have two pairs of induction hypotheses:
		\begin{enumerate}[1$_\phi$.]
	
		\item if $\phi\in B$, then $\llbracket\phi\rrbracket_{v_B}=1$, and 
		\item if $\neg \phi\in B$, then $\llbracket\phi\rrbracket_{v_B}=0$.
	
	\end{enumerate}
	
	And
	
	\begin{enumerate}[1$_\psi$.]
	
		\item if $\psi\in B$, then $\llbracket\psi\rrbracket_{v_B}=1$, and 
		\item if $\neg \psi\in B$, then $\llbracket\psi\rrbracket_{v_B}=0$.
	
	\end{enumerate}
	What we need to prove are:
	\begin{enumerate}[1$_{\phi\land\psi}$.]
	
		\item if $\phi\land \psi\in B$, then $\llbracket\phi\land \psi\rrbracket_{v_B}=1$, and 
		\item if $\neg (\phi\land \psi)\in B$, then $\llbracket\phi\land \psi\rrbracket_{v_B}=0$.
	\end{enumerate}

		We do so in turn:
		
		\begin{itemize}
		
			\item Suppose that $\phi\land \psi\in B$. Since $B$ is an open branch of a complete tableau, every rule that can be applied has been applied. And so, the rule for $\phi\land \psi$ has been applied:
			\begin{center}
				\begin{prooftree}
					{
					line numbering=false,
					line no sep= 2cm,
					for tree={s sep'=10mm},
					single branches=true,
					close with=\xmark
					}
					[\phi\land\psi [\phi [\psi ] ] ]
					\end{prooftree}
				\end{center}
		So, we can conclude that both $\phi,\psi\in B$. But by 1$_\phi$. and 1$_\psi$., this means that $\llbracket\phi\rrbracket_{v_B}=1$ and $\llbracket\psi\rrbracket_{v_B}=1$. Since $\llbracket\phi\land \psi\rrbracket_{v_B}=min(\llbracket\phi\rrbracket_{v_B},\llbracket\psi\rrbracket_{v_B})$, we get $\llbracket\phi\land \psi\rrbracket_{v_B}=1,$ as desired.
		
		
		\item Next, suppose that $\neg (\phi\land \psi)\in B$. Again, since $B$ is an open branch of a complete tableau, the rule for $\neg(\phi\land \psi)$ has been applied:
		\begin{center}
				\begin{prooftree}
					{
					line numbering=false,
					line no sep= 2cm,
					for tree={s sep'=10mm},
					single branches=true,
					close with=\xmark
					}
					[\neg(\phi\land\psi) [\neg\phi ] [\neg\psi ] ]
					\end{prooftree}
				\end{center}
		
		\end{itemize}
	So, we can conclude that either $\neg\phi\in B$ or $\neg\psi\in B$. We can therefore distinguish two cases. In the first case, if $\neg\phi\in B$, we can infer from 2$_\phi$. that $\llbracket \phi\rrbracket=0$. Since $\llbracket\phi\land \psi\rrbracket_{v_B}=min(\llbracket\phi\rrbracket_{v_B},\llbracket\psi\rrbracket_{v_B})$, this means we get $\llbracket\phi\land \psi\rrbracket_{v_B}=0$. In the second case, if $\neg\psi\in B$, we can infer from 2$_\psi$. that $\llbracket \phi\rrbracket=0$. So, we get $\llbracket\phi\land \psi\rrbracket_{v_B}=0$. So either way,  $\llbracket\phi\land \psi\rrbracket_{v_B}=0$---as desired.
		
		\end{enumerate}
	
	\end{enumerate}
			As we said above, the remaining cases are left as exercises. Once they are completed, we can infer the completeness lemma via induction.
	\end{proof}

	\item From the completeness lemma, actual completeness follows quickly (via the proof strategy laid out in 7.3.1):
	\begin{theorem}[Propositional Completeness]
	If $\Gamma\vDash\phi$, then $\Gamma\vdash\phi$.
	\end{theorem}
	
	\begin{proof}
	We use contrapositive proof. So assume that $\Gamma\nvdash\phi$. This means, by definition, that in the complete tableau for $\Gamma\cup\{\neg\phi\}$ there is at least one open branch. Call it $B$. The branch has an associated interpretation $v_B$ which is faithful to $B$ by Lemma 7.3.2. Since $\Gamma\cup\{\neg\phi\}$ is the root of the tableau, and therefore included in any branch, we have $\Gamma\cup\{\neg\phi\}\subseteq B$. But that means, sine $v_B$  is faithful to $B$, that $v_B$ makes all the members of $\Gamma\cup\{\neg\phi\}$ true and so $\Gamma\nvDash\phi$ (by 6.2.6), as desired.
	\end{proof}
	
	\item Note that, just like in the case of soundness, our completeness theorem can be interpreted as a theorem about the tableau method with respect to satisfiability search: we've effectively shown that the method works with respect to satisfiability.

	\begin{theorem}[Tableau Verification Part 2.] If the algorithm laid out in 6.3.2 gives the answer that a set is satisfiable, then the set \emph{is} satisfiable.
	\end{theorem}
	\begin{proof}
	Let $\Gamma$ be a set the algorithm says is satisfiable. This means that the complete tableau for $\Gamma$ is open. So, there is an open branch, $B$, in the tableau with an associated valuation $v_B$. By Lemma 7.3.2, $v_B$ is faithful to $B$ and since $\Gamma\subseteq B$, it follows that $v_B$ makes all the members of $\Gamma$ is true. In other words, $\Gamma$ is satisfiable.
	\end{proof}
	
	\item Together with the observation that the tableau method applied to a finite set terminates after finitely many steps, Theorems 7.2.3 and 7.3.4 give an alternative proof of Theorem 5.3.7, the decidability of classical propositional logic:
	\begin{theorem}[Decidability of Propositional Logic]
	Propositional logic is decidable, i.e. there exists an algorithm which after finitely many steps correctly determines whether a given inference (with finitely any premises) is valid.
	\end{theorem}
	But we already know that propositional logic is decidable, via truth-tables, so why do we need \emph{another} proof? The answer reveals something important about mathematical practice. Let's assume that we're not worried about any of our proofs being incorrect and looking for confirmation of the result by repeated proofs. This would not be good mathematical practice anyways: remember we work rigorously! Rather, having an alternative proof of an already known result can provide new \emph{insight} into the result: why it is true, what it says, how it can be applied, etc. In the case of decidability, we have an alternative proof via tableaux. This is desirable since we already know that many other logics (like first-order logic but also any non-classical logics) don't have (something corresponding to) a truth-table method. So, if we want to prove decidability for these other logics, we can't use truth-tables. But, we can try tableaux! It is, in fact, possible to develop tableau methods for a wide range of logics (you will see tableaux for other logics throughout your degree). If we can develop a sound and complete tableau method for another logic, we can at least hope that we get decidability in this way. Unfortunately, in the case of first-order logic, our hopes will be disappointed. But even in the \emph{failure} to obtain decidability for first-order logic lies some insight: we'll be able to see \emph{see} why first-order logic is undecidable.		
	
\end{enumerate}
		
\section{Infinite Premiss Sets and Compactness}

\begin{em}
  In this section, we remain a bit less mathematically precise than we usually are in this course.
  This is because we haven't introduced the proper methods of dealing with infinity and the topic is infinity in logic.
  However, we will introduce some ideas that will become important in first-order logic, so stay tuned.
\end{em}

	\begin{enumerate}[\thesection.1]

	  \item We conclude our treatment of tableaux and of propositional logic in general by looking at inferences with \emph{infinite} premise sets.
		So far, especially for tableaux and truth-tables, we've restricted ourselves to inferences with finite premise sets.
		And for good reason: humans usually don't make arguments with infinitely many statements in them; how could they?
		But, especially from a mathematical perspective, it's desirable to be able to deal with inferences potentially involving infinite premises.
		It turns out that many mathematical theories need to have \emph{infinitely many axioms}.
		The standard theory of the natural numbers, called \emph{Peano arithmetic} typically abbreviated $PA$, is already an example.
		It's in fact provable that there is no finite set of axioms that describes the same theory as $PA$ (although we will not prove this here).
		So, already for something as simple as a proper mathematical treatment of $0,1,2,\mathellipsis$ we need inferences with infinite premise sets.
		We'll get familiar with $PA$ in the following section, $PA$ is a \emph{first-order theory}.
		For now, we'll simply discuss how we can make tableau work for inferences with infinitely many premises in the ``safe harbor'' of propositional logic.
		
		\item We'll, in fact, still make a limiting assumption, namely that the infinite premise sets we'll be considering can be \emph{indexed} or \emph{enumerated} by the (positive) natural numbers. That is, if $\Gamma$ is a premise set, then there is a way of writing $\Gamma$ as the set $\{\phi_i:i\in I\}$, where $I\subseteq\mathbb{N}^+$;\footnote{$\mathbb{N}^+=\mathbb{N}\setminus\{0\}$.} in other words, there is a first member of $\Gamma$, a second member of $\Gamma$, and so on.\footnote{This doesn't mean that there's an \emph{algorithm} that does that; this is a different, much more complicated question.}  It might be surprising to learn that there are actually infinite sets of formulas \emph{bigger than that}. And, in fact, they only occur in the more technical realms of logic. Even in most infinitary applications, our premise sets can still be indexed. In fact, coming from a computer science perspective, the assumption that our premise sets are enumerable is quite reasonable from a technical perspective: computers \emph{can} handle infinity, like the natural numbers, using inductive definitions and recursion; but with respect to the higher-realms of infinity, computers are rather limited in their capabilities (computers already approximate reals like $\pi$ with \emph{floats}).
		
		\item There are many, many infinite premise sets you can imagine, here we give just a few examples to show you what we're dealing with:
		
		\begin{enumerate}[(a)]
		
			\item The set $\{p, \neg p, \neg\neg p, \mathellipsis\}$ or more precisely the smallest set $X$ such that $p\in X$ and if $\phi\in X$ then $\neg \phi\in X$. To have this infinite set, we don't even need infinitely many sentence letters.
			
			\item But if we do, the set $\mathcal{P}=\{p_i: i\in \mathbb{N}\}$ can function as an infinite premise set.
			
			\item So, we can also have a set like this $\{\neg p_{2i}, p_{2i+1}: i\in\mathbb{N}\}$ which contains $\neg p_i$ for each even $i$ and $p_i$ for each odd $i\in\mathbb{N}$.
			
			\item Let $v$ be any valuation. Then the set $T_v=\{\phi:\llbracket\phi\rrbracket_v=1\}$ is always infinite! This idea we'll discuss in more detail in the context of first-order logic. But to see this, suppose that $v(p)=1$ (if $v(p)=0$, the argument is completely analogous). Then $\llbracket p\rrbracket_v=1$ and so $p\in T_v$. But also $\llbracket \neg\neg p\rrbracket_v=1$ and so $\neg\neg p\in T_v$. And note that $p\neq \neg\neg p$---after all, $p$ contains no negations and $\neg\neg p$ contains 2. Hence, $T_v$ has at least two members. But then, there's also $\neg\neg\neg\neg p$, $\llbracket \neg\neg\neg\neg p\rrbracket_v=1$ so $\neg\neg\neg\neg p\in T_v$, and $p\neq \neg\neg\neg\neg p, \neg\neg p\neq \neg\neg\neg\neg p$---so $T_v$ has at least 3 members. This clearly goes on, so for every $n$, $T_v$ has at least $n$ members, which is just another way of saying that $T_v$ is an infinite set.
		
		\end{enumerate}
		
		\item Note that the definition of logical consequence/validity given in 5.2.2 can easily be applied to cases with infinitely many premises:
				\begin{itemize}
		
			\item $\Gamma\vDash\phi$ iff for all valuations $v$, if $\llbracket\psi\rrbracket_v=1$, for all $\psi\in\Gamma$, then $\llbracket\phi\rrbracket_v=1$.
		
		\end{itemize}
		The definition requires that under all valuations where all the premises are true, also the conclusion is true. There's nothing finitary going on here: even if there are infinitely many premises, they can all be true under a valuation.
		
		\item Where it get's tricky is if we want to use truth-tables to determine validity. Unfortunately, the method no longer works. Remember from 5.3.6 that in order to check whether $\phi_1,\mathellipsis, \phi_n\vDash \psi$ we did the truth-table for $(\phi_1\land\mathellipsis\land\phi_n)\to\psi$ (using result 5.2.16 that $\phi_1,\mathellipsis, \phi_n\vDash \psi$ iff $\vDash (\phi_1\land\mathellipsis\land\phi_n)\to\psi$ in the background). But if our premises are infinite, this idea breaks down. It is not possible to form a conditional with an infinite conjunction of premises as the if-part. And even if we could (which we can in \emph{infinitary logic}), we'd still have the problem that we can't always write down the truth-table for such a conditional: if we have infinitely many premises, we can have infinitely many sentence letters involved, and as we remarked in 5.1.4, we simply can't list all the possible distributions of truth-values on an infinite set of sentence letters. So, once we ``go infinitary,'' truth-tables are out. 
		
	  \item Fortunately, tableaux are not.
		We'll now describe how the tableau method works for checking whether
		$\Gamma\vDash\phi$
		when
		$\Gamma$
		is infinitary.
		For our algorithm in 7.3.2, we began by writing down
		$\Gamma\cup\{\neg\phi\}$
		as the initial list.
		This no longer works,
		since $\Gamma$ is infinite.
		So, instead, in our first step,
		what we're going to do is to write down
		$\neg\phi$
		as our initial list.
		Then,
		without considering the premises for,
		we simply repeatedly apply the tableau rules to make the complete tableau for
		$\neg\phi$.
		If this tableau closes, we can already declare that
		$\Gamma\vdash\phi$,
		since if the tableau for
		$\neg\phi$
		closes, this means that
		$\phi$
		is valid,
		i.e. $\emptyset\vDash\phi$,
		and by Monotonicity (cf. 5.2.6) if follows that
		$\Gamma\vDash\phi$.
		So, let's continue assuming that the tableau for $\neg\phi$ doesn't close.
		Now, it will be important that we can write $\Gamma$ as
		$\{\psi_i:i\in I\}$,
		where $I\subseteq\mathbb{N}^+$,
		or, more transparently, as
		$\{\psi_1, \psi_2, \mathellipsis\}$.
		Now,
		what we're going to do is to write $\psi_1$ into the initial list,
		which so far only contained
		$\neg\phi$.
		Then we repeatedly apply the rules again to the new tableau.
		If the tableau closes, we declare that
		$\Gamma\vdash\phi$,
		since we've just shown that
		$\psi_1\vDash\phi$
		(we did the tableau for $\{\psi_1,\neg\phi\}$)
		and by Monotonicity,
		$\psi_1, \psi_2, \mathellipsis\vDash\phi$
		follows.
		If the tableau still doesn't close, we repeat this procedure with $\psi_2$, then with $\psi_3,$ and so on.
		If at any step the tableau closes, we claim $\Gamma\vdash\phi$.
		If we continue going through the $\psi_i$'s and the tableau never closes,
		we declare $\Gamma\nvdash\phi$.
		
		\item Note that if $\Gamma\nvdash \phi$, we might never actually get the job done---we might continue checking for an infinite amount of time (well not actually, but we could continue indefinitely). But this doesn't mean that we can't show that $\Gamma\nvdash\phi$. It could happen, for example, that at some point we can \emph{prove} that the tableau will never close. This would require an insight concerning the structure of the premise set. For example, if $\Gamma$ is the set $\{\underbrace{\neg \mathellipsis\neg}_{n\text{ times}}p: n\in\mathbb{N}\}$\footnote{Or more precisely, the smallest set $X$ such that $p\in X$ and if $\phi\in X$, then $\neg\neg\phi\in X$.} If the tableau for $\Gamma\cup\{\neg\phi\}$ doesn't close after the first step, i.e. the tableau for $\{p,\neg\phi\}$, we can already see that the tableau will never ever close: after the initial step, all that our procedure will do is to add a new version of $\neg\neg p$ to the initial list, apply the rule for $\neg\neg p$ to give us a new node with $p$, then adding $\neg\neg \neg\neg p$ to the list, which gives us a new node with $\neg\neg p$ and then a new node with $p$, and so on. Basically, the only new nodes with sentence letters we can ever get from the premise set $\Gamma$ contain $p$; and so if the tableau for $\{p,\neg\phi\}$ doesn't close, we will never be able to close.
		
		\item A consequence of the possibility that a tableau will never close is that we will not get decidability for inferences with infinite premise sets. Remember from the introduction (\S1.4) that for decidability, we require an algorithm with \emph{finite} run time. And it might just happen that our algorithm just described keeps on running forever. This is bad, since we'll never be able to know if the algorithm keeps running because it hasn't closed \emph{yet} or because it never will. So, we can't use our algorithm to \emph{effectively} determine whether a given infinitary inference is valid. 
	
		\item It is, however, still possible to prove soundness and completeness for the infinitary tableau method just described, though we won't do this here.\footnote{Basically, the argument is the same as in the finitary case we discussed above, we just need to be a bit careful with the infinities that might occur.} This might be surprising but it's important to see that decidability and soundness and completeness are not the same thing. That we still get soundness, is not so surprising, in fact. Note that for soundness, what matters is that if our algorithm says that a set is unsatisfiable, then it is unsatisfiable. But the algorithm saying that a set is unsatisfiable just means that the tableau closes. And if a tableau closes it closes after finitely many steps (there will have to be a point at which all branches have been closed). What might be a bit more surprising is that we still get completeness: what we needed to show for completeness is that the associated valuation for a branch is faithful to that branch. But what if the branch is infinite? Well, note that the definition of $v_B$ actually doesn't require that $B$ is finite (6.3.5):
		\[v_B(p):=\begin{cases} 1 &\text{if }p\in B\\0&\text{if }p\notin B\end{cases}\]	
		This definition works perfectly fine even if $B$ is infinite. The rest is just details (albeit infinitary details).
		
			\item We will conclude the chapter by proving an interesting theorem, which relies on the soundness and completeness of the infinitary method, but, in a sense, makes the method obsolete:
	\begin{theorem}[Compactness]
	Let $\Gamma$ be an infinite set of formulas. If $\Gamma\vdash\phi$, then there exists a finite set $\Sigma\subseteq\Gamma$ such that $\Sigma\vdash\phi$. 
	\end{theorem}	
	\begin{proof}
	As always, we assume that $\Gamma$ can be written $\{\psi_i: i\in I\}$ for some $I\subseteq\mathbb{N}^+$. Now suppose that $\Gamma\vdash\phi$. This means that the tableau for the infinitary inference closes at some point. Suppose that the last formula we added to the initial list is $\psi_n$. This means that the tableau for $\{\psi_1, \mathellipsis, \psi_n,\neg\phi\}$ closes (so far, all we did \emph{is} to construct the tableau for this set). But that just means that $\psi_1, \mathellipsis,\psi_n\vdash\phi$. Since $\{\psi_1, \mathellipsis,\psi_n\}\subseteq\Gamma$, our claim holds.
	\end{proof}
	The compactness theorem states that if there is a proof from an infinite premise set, then there's always a proof from a finite subset. This is encouraging, we don't really \emph{need} the infinitary methods just described to prove the things we want to prove.\footnote{In light of this result, my claim from 7.4.1 that $PA$ can't be finitely axiomatized might be confusing. But note that all we've shown is that for every number theoretic fact there's a finite set of premises that suffices to show it. But for two different facts, these sets of premises might be different. In fact, if we want to have a set of premises that allows us to derive \emph{all} number theoretic facts, this set needs to be infinite.} Well, the theorem will become \emph{really} interesting in first-order logic. For now, we rest content with having achieved its proof for propositional logic.
			
	\end{enumerate}
					
\section{Core Ideas}

\begin{itemize}

	\item By proving soundness and completeness we're reducing validity to syntax.
	
	\item The soundness theorem is a kind of sanity check for a proof system and typically easier to prove.
	
	\item The completeness theorem is a surprising mathematical fact and typically harder to prove.
	
	\item In the case of tableaux, the soundness theorem relies on the fact that in every rule, if the upper formula is true, then at least one of the lower formulas is true (`down preservation').
	
	\item In the case of tableaux, the completeness theorem relies on the fact that if one of the lower formulas in a rule is true, so is the upper formula (`up preservation').
	
	\item Truth tables don't work for infinitary premise sets but tableaux do.
	
	\item Compactness tells us that if there's a proof, we can always find a finitary one.

\end{itemize}

\section{Self Study Questions}

	\begin{enumerate}[\thesection.1]
	
		\item Which of the following entails that your proof system is \emph{un}sound? (Assume that you're trying to develop a proof system for classical logic).
		
		\begin{enumerate}[(a)]
		
			\item There is a valid inference in which you can't derive the conclusion from the premises.
			
			\item There is an invalid inference in which you can derive the conclusion from the premises.
			
			\item There is a set of premises from which you can both derive a formula and it's negation.
			
			\item There is a satisfiable set of premises from which you can derive every formula whatsoever.
					
		\end{enumerate}
				
			\item Which of the following entails that your proof system is \emph{in}complete? (Assume that you're trying to develop a proof system for classical logic).
		
		\begin{enumerate}[(a)]
		
			\item There is a valid inference in which you can't derive the conclusion from the premises.
			
			\item There is an invalid inference in which you can derive the conclusion from the premises.
			
						\item There is a set of premises from which you can't derive any formula whatsoever.
			
			\item There is a set of premises from which you can both derive a formula and it's negation.
			
					
		\end{enumerate}		
				
	\end{enumerate}

\section{Exercises}

	\begin{enumerate}[\thesection.1]
	
		\item Check the remaining cases of Proof 7.2.4.
		
		\item Check the remaining cases of Proof 7.3.2.
	
		\item $[h]$ Let's say that a set of formulas $\Gamma$ is \emph{proof-theoretically inconsistent} iff there exists a formula $\phi$ such that $\Gamma\vdash\phi$ and $\Gamma\vdash\neg\phi$. Correspondingly, we say that $\Gamma$ is \emph{proof theoretically consistent} iff there is no formula $\phi$ such that both $\Gamma\vdash\phi$ and $\Gamma\vdash\neg\phi$. 
		
		\begin{enumerate}
		
		\item Use the soundness theorem to derive that every proof-theoretically inconsistent set is unsatisfiable.
		
		\item Use the completeness theorem to derive that every proof-theoretically consistent set is satisfiable.
		
		\end{enumerate}
		
		\item Let $\Gamma$ be a set of formulas such that there exists a formula $\phi$ with $\Gamma\nvdash\phi$. Use the completeness theorem to conclude that $\Gamma$ is satisfiable.
		
		%\item $[h]$ Let $\Gamma$ be a satisfiable set of formulas. Use the soundness theorem to conclude that for all formulas $\phi$, either $\Gamma\nvdash\phi$ or $\Gamma\nvdash \neg\phi$.
		
		\item Suppose that $\{\phi\}$ is satisfiable and $\phi\vdash\psi$. Use the soundness theorem to conclude that $\psi\nvdash\neg\phi$.
						
	\end{enumerate}

\vfill

\hfill \rotatebox[origin=c]{180}{
\fbox{
\begin{minipage}{0.5\linewidth}

\subsection*{Self Study Solutions}

\begin{itemize}

	\item[7.6.1] (b), (d)
	
	\item[7.6.2] (a), (c)

\end{itemize}


\end{minipage}}}

%%% Local Variables: 
%%% mode: latex
%%% TeX-master: "../../logic.tex"
%%% End:
